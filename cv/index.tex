% Options for packages loaded elsewhere
% Options for packages loaded elsewhere
\PassOptionsToPackage{unicode}{hyperref}
\PassOptionsToPackage{hyphens}{url}
\PassOptionsToPackage{dvipsnames,svgnames,x11names}{xcolor}
%
\documentclass[
  letterpaper,
  DIV=11,
  numbers=noendperiod]{scrartcl}
\usepackage{xcolor}
\usepackage{amsmath,amssymb}
\setcounter{secnumdepth}{5}
\usepackage{iftex}
\ifPDFTeX
  \usepackage[T1]{fontenc}
  \usepackage[utf8]{inputenc}
  \usepackage{textcomp} % provide euro and other symbols
\else % if luatex or xetex
  \usepackage{unicode-math} % this also loads fontspec
  \defaultfontfeatures{Scale=MatchLowercase}
  \defaultfontfeatures[\rmfamily]{Ligatures=TeX,Scale=1}
\fi
\usepackage{lmodern}
\ifPDFTeX\else
  % xetex/luatex font selection
\fi
% Use upquote if available, for straight quotes in verbatim environments
\IfFileExists{upquote.sty}{\usepackage{upquote}}{}
\IfFileExists{microtype.sty}{% use microtype if available
  \usepackage[]{microtype}
  \UseMicrotypeSet[protrusion]{basicmath} % disable protrusion for tt fonts
}{}
\makeatletter
\@ifundefined{KOMAClassName}{% if non-KOMA class
  \IfFileExists{parskip.sty}{%
    \usepackage{parskip}
  }{% else
    \setlength{\parindent}{0pt}
    \setlength{\parskip}{6pt plus 2pt minus 1pt}}
}{% if KOMA class
  \KOMAoptions{parskip=half}}
\makeatother
% Make \paragraph and \subparagraph free-standing
\makeatletter
\ifx\paragraph\undefined\else
  \let\oldparagraph\paragraph
  \renewcommand{\paragraph}{
    \@ifstar
      \xxxParagraphStar
      \xxxParagraphNoStar
  }
  \newcommand{\xxxParagraphStar}[1]{\oldparagraph*{#1}\mbox{}}
  \newcommand{\xxxParagraphNoStar}[1]{\oldparagraph{#1}\mbox{}}
\fi
\ifx\subparagraph\undefined\else
  \let\oldsubparagraph\subparagraph
  \renewcommand{\subparagraph}{
    \@ifstar
      \xxxSubParagraphStar
      \xxxSubParagraphNoStar
  }
  \newcommand{\xxxSubParagraphStar}[1]{\oldsubparagraph*{#1}\mbox{}}
  \newcommand{\xxxSubParagraphNoStar}[1]{\oldsubparagraph{#1}\mbox{}}
\fi
\makeatother


\usepackage{longtable,booktabs,array}
\usepackage{calc} % for calculating minipage widths
% Correct order of tables after \paragraph or \subparagraph
\usepackage{etoolbox}
\makeatletter
\patchcmd\longtable{\par}{\if@noskipsec\mbox{}\fi\par}{}{}
\makeatother
% Allow footnotes in longtable head/foot
\IfFileExists{footnotehyper.sty}{\usepackage{footnotehyper}}{\usepackage{footnote}}
\makesavenoteenv{longtable}
\usepackage{graphicx}
\makeatletter
\newsavebox\pandoc@box
\newcommand*\pandocbounded[1]{% scales image to fit in text height/width
  \sbox\pandoc@box{#1}%
  \Gscale@div\@tempa{\textheight}{\dimexpr\ht\pandoc@box+\dp\pandoc@box\relax}%
  \Gscale@div\@tempb{\linewidth}{\wd\pandoc@box}%
  \ifdim\@tempb\p@<\@tempa\p@\let\@tempa\@tempb\fi% select the smaller of both
  \ifdim\@tempa\p@<\p@\scalebox{\@tempa}{\usebox\pandoc@box}%
  \else\usebox{\pandoc@box}%
  \fi%
}
% Set default figure placement to htbp
\def\fps@figure{htbp}
\makeatother





\setlength{\emergencystretch}{3em} % prevent overfull lines

\providecommand{\tightlist}{%
  \setlength{\itemsep}{0pt}\setlength{\parskip}{0pt}}



 


\KOMAoption{captions}{tableheading}
\makeatletter
\@ifpackageloaded{caption}{}{\usepackage{caption}}
\AtBeginDocument{%
\ifdefined\contentsname
  \renewcommand*\contentsname{Table of contents}
\else
  \newcommand\contentsname{Table of contents}
\fi
\ifdefined\listfigurename
  \renewcommand*\listfigurename{List of Figures}
\else
  \newcommand\listfigurename{List of Figures}
\fi
\ifdefined\listtablename
  \renewcommand*\listtablename{List of Tables}
\else
  \newcommand\listtablename{List of Tables}
\fi
\ifdefined\figurename
  \renewcommand*\figurename{Figure}
\else
  \newcommand\figurename{Figure}
\fi
\ifdefined\tablename
  \renewcommand*\tablename{Table}
\else
  \newcommand\tablename{Table}
\fi
}
\@ifpackageloaded{float}{}{\usepackage{float}}
\floatstyle{ruled}
\@ifundefined{c@chapter}{\newfloat{codelisting}{h}{lop}}{\newfloat{codelisting}{h}{lop}[chapter]}
\floatname{codelisting}{Listing}
\newcommand*\listoflistings{\listof{codelisting}{List of Listings}}
\makeatother
\makeatletter
\makeatother
\makeatletter
\@ifpackageloaded{caption}{}{\usepackage{caption}}
\@ifpackageloaded{subcaption}{}{\usepackage{subcaption}}
\makeatother
\usepackage{bookmark}
\IfFileExists{xurl.sty}{\usepackage{xurl}}{} % add URL line breaks if available
\urlstyle{same}
\hypersetup{
  colorlinks=true,
  linkcolor={blue},
  filecolor={Maroon},
  citecolor={Blue},
  urlcolor={Blue},
  pdfcreator={LaTeX via pandoc}}


\author{}
\date{}
\begin{document}

\renewcommand*\contentsname{Table of contents}
{
\hypersetup{linkcolor=}
\setcounter{tocdepth}{3}
\tableofcontents
}

\section{Summary}\label{summary}

I am an R developer and bioinformatician with \textgreater{} 6 years
delivering data-driven solutions.

\section{Skills}\label{skills}

\subsection{Programming}\label{programming}

\begin{itemize}
\tightlist
\item
  R
\item
  Python
\item
  SQL
\item
  JavaScript
\item
  Bash
\end{itemize}

\subsection{Bioinformatics}\label{bioinformatics}

\begin{itemize}
\tightlist
\item
  DNA and RNA sequencing
\item
  ASO design
\item
  Proteomics
\item
  High-throughput screening development
\end{itemize}

\subsection{Data science}\label{data-science}

\begin{itemize}
\tightlist
\item
  Tidyverse
\item
  Quarto and Rmarkdown
\item
  Machine learning
\end{itemize}

\subsection{Engineering}\label{engineering}

\begin{itemize}
\tightlist
\item
  Git
\item
  Docker
\item
  CI/CD
\item
  NextFlow
\item
  Linux
\item
  Azure
\item
  Relational database design
\end{itemize}

\section{Experience}\label{experience}

\subsection{Audit Scotland, Glasgow.}\label{audit-scotland-glasgow.}

\subsubsection{R/R Shiny
Developer2025-Present}\label{rr-shiny-developer2025-present}

\begin{itemize}
\item
  I design and maintain web-based data applications (predominately using
  R but also Python) that are used by teams of financial and performance
  auditors to perform a range of analyses. These include but are not
  limited to:

  \begin{itemize}
  \item
    Analysis of general ledger and financial statement data.
  \item
    Collation of Audit Scotland parliamentary interaction data.
  \item
    Investigation of NHS Scotland and local authority performance metric
    data.
  \end{itemize}
\item
  I design and maintain data pipelines for extracting, transforming and
  loading public sector data into structured and semi-structured data
  sets such as relational databases or JSON file stores, respectively.
\item
  I use Git for version control and develop GitHub pipelines for
  maintaining code and adhering continuous integration (CI) and
  continuous development (CD) practices.
\item
  I lead the development of best practicee guidance documentation for R
  programming at Audit Scotland.
\end{itemize}

\subsection{Fios Genomics, Edinburgh}\label{fios-genomics-edinburgh}

\subsubsection{Senior Bioinformatics
Developer2023-2025}\label{senior-bioinformatics-developer2023-2025}

\begin{itemize}
\item
  I contributed to an internal codebase using R, SQL, Python and
  JavaScript that was used by a team of bioinformaticians. Typical
  analysis included but was not limited to:

  \begin{itemize}
  \item
    Critical and confounding factors analysis biological experimental
    study designs.
  \item
    Analysis of high dimenstional data such as gene expression,
    proteomic or custom array data sets. This include exploratory
    analyses using techniques such as clustering and dimension reduction
    as well as statistical analyses using models such as linear models
    implemented in the ``limma'' R package.
  \item
    Hypergeometric and gene-set enrichment testing of biological
    pathways such as Reactome pathways and GO terms.
  \item
    Designing personalised allele-specific anti-sense oligonucleotide
    (ASO) using long read PacBio sequencing data.
  \item
    Data mining of public bioinformatic databases such as gene
    expression omnibus (GEO) and the cancer genome atlas (TCGA).
  \item
    Development of machine learning models for predicting patient drug
    response using omics data.
  \end{itemize}
\item
  I was involved in the communication of code updates and new features
  via blog posts and presentations to a team of bioinformaticians.
\item
  I was responsible for ensuring reproducibility of statistical analyses
  and robustness of internal codebase using tools such as Docker and
  GitLab CI pipelines.
\end{itemize}

\subsubsection{Senior
Bioinformatician2021-2023}\label{senior-bioinformatician2021-2023}

\begin{itemize}
\item
  I acted as the lead bioinformatic data scientist on a range of
  different statistical analysis projects.
\item
  I provided scientific and technical guidance to junior
  bioinformaticians.
\item
  I was regularly involves in scoping statistical analysis plans for new
  and existing clients.
\end{itemize}

\subsubsection{Bioinformatician2021-2023}\label{bioinformatician2021-2023}

\begin{itemize}
\tightlist
\item
  I acted as a junior bioinformatic data scientist, performing
  statistical analyses for clients in the pharmaceutical and academic
  sectors.
\end{itemize}

\subsection{Cancer Research UK Scotland Institute,
Glasgow}\label{cancer-research-uk-scotland-institute-glasgow}

\subsubsection{Laboratory Aide2015-2016}\label{laboratory-aide2015-2016}

\begin{itemize}
\tightlist
\item
  I was part of a large team responsbile for sterilisation of reusable
  laboratory glassware, preparation of tissue culture solutions, and
  autoclaving of laboratory waste.
\end{itemize}

\section{Education}\label{education}

\subsection{University of Glasgow}\label{university-of-glasgow}

\subsubsection{Master of Science - Bioinformatics with
Distinction2018-2019}\label{master-of-science---bioinformatics-with-distinction2018-2019}

\begin{itemize}
\item
  Classes included foundations in bioinformatics, Java and R
  programming, and database theory.
\item
  Awarded Scottish Funding Council scholarship based on undergraduate
  academic performance.
\end{itemize}

\subsection{University of Strathclyde}\label{university-of-strathclyde}

\subsubsection{Bachelor of Science - Pharmacology and Biochemistry with
First Class
Honours2016-2018}\label{bachelor-of-science---pharmacology-and-biochemistry-with-first-class-honours2016-2018}

\begin{itemize}
\item
  Classes included clinical and advanced biochemistry and pharmacology,
  and laboratory skills.
\item
  Awarded Biochemical Society prize for academic achievement.
\end{itemize}

\subsection{University of Glasgow}\label{university-of-glasgow-1}

\subsubsection{Diploma of Higher Education - Medical
Science2013-2015}\label{diploma-of-higher-education---medical-science2013-2015}

\begin{itemize}
\tightlist
\item
  Classes included anatomy, physiology and pharmacology.
\end{itemize}

\subsection{The Glasgow Academy}\label{the-glasgow-academy}

\begin{itemize}
\tightlist
\item
  6 SQA Higher Grades and 2 SQA Advanced Higher Grades -- all at Grade
  A.
\end{itemize}




\end{document}
